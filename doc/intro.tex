%%%%%%%%%%%%%%%%%%%%%%%%%%%%%%%%%%%%%%%%%%%%%%%%%%%%%%%%%%%%%%%%%%%%%%%%%
%%
%W  intro.tex             FORMAT documentation    B. Eick and C.R.B. Wright
%%
%% 10-30-00

%%%%%%%%%%%%%%%%%%%%%%%%%%%%%%%%%%%%%%%%%%%%%%%%%%%%%%%%%%%%%%%%%%%%%%%%%
\def\FORMAT{{\sf FORMAT}}
\def\F{{\cal{F}}}
\Chapter{Introduction to FORMAT}

\index{Format}

The GAP 4 package {\FORMAT} provides functions to compute with formations 
of finite solvable groups.  In addition to tools for constructing and 
combining formations, the package contains functions to compute 
$\F$-residual subgroups and to construct $\F$-normalizers and 
$\F$-covering subgroups determined by locally defined formations. 
System normalizers and Carter subgroups are available as special cases, 
and the $\F$-normalizer functions also apply to the computation of 
complements. The corresponding algorithms, together with applications 
and a complexity analysis, are described in~\cite{EW}. 

The package permits the computation of formation-theoretic subgroups 
not only for a number of classical formations, such as nilpotent,
supersolvable or $p$-length 1 groups, but for other formations that the
user may define. It also allows computation with classes of
finite solvable groups defined by normal subgroup functions (see
\cite{DH}, pages 395~ff). Attention may be restricted to the
subgroups of a single group, a feature that has applications
in the computation of complements to elementary abelian normal subgroups
in finite solvable groups (see \cite{EW}). An example of such an 
application is given in Section~"Other Applications".

This documentation contains only a brief account of the main 
formation-theoretic ideas. For a much more complete treatment we
refer the reader to \cite{DH}. Fundamental ideas of formation theory are
described in \cite{G} and \cite{CH}.

In the following sections we first describe the {\GAP} definition of a
formation and the examples of standard formations that are included in 
the package. We also present some functions that obtain new formations 
from ones already defined or that modify defined formations slightly.
(See Section~"Formations in GAP".) 

Then we describe functions that compute formation-theoretic subgroups 
of finite solvable groups (see Sections "Residual Functions", 
"FNormalizers" and~"Covering Subgroups"). 

Finally we provide examples from a {\GAP} session (see Sections~"Formation
Examples" and "Other Applications") to illustrate the functions in the package.

